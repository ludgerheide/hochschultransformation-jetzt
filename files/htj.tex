% Print
\documentclass[DIV=12]{scrartcl}

%Für Tablets:
%\documentclass{scrartcl}
%\usepackage[margin=5mm,a5paper,includefoot]{geometry}

%Packages, die für die deutsche Sprache erforderlich sind
\usepackage[utf8]{inputenc}
\usepackage[T1]{fontenc}
\usepackage{lmodern}
\usepackage[ngerman]{babel}

%Packages für Graphik
\usepackage{graphicx}
\graphicspath{{figures/}}

%BibLaTex
%\usepackage[backend=biber]{biblatex}
%\addbibresource{../Literatur/Literaturliste.bib} 
\usepackage{csquotes}

%Package, damit Bibtex-URL klappt
\usepackage{hyperref}
\usepackage{url}

\usepackage{parskip}

\usepackage{draftwatermark}
\SetWatermarkText{Entwurf}
\SetWatermarkScale{1}

%Package für schöne Tabellen mit variabler Breite
%\usepackage{tabulary}
%\usepackage{booktabs}

%Noch schönere Typographie
\usepackage{microtype}

%Abschnittsnummer verstecken
%\renewcommand*{\sectionformat}{}

%%%%% BEGINN KOPF- UND FUẞZEILE %%%%%
\usepackage[footsepline]{scrlayer-scrpage}
\pagestyle{scrheadings}
\clearscrheadfoot

\ofoot{\pagemark}
%\chead{\small{Fakultät V -- Verkehrs- und Maschinensysteme\\
%Methoden der Produktentwicklung und Mechatronik\\
%Prof. Dr.-Ing. Dietmar Göhlich}}
\ifoot{Hochschultransformation jetzt!}
%\cfoot{\pagemark}
%\ofoot{\today}
%%%%% ENDE KOPF- UND FUẞZEILE %%%%%

%Roboto als Schriftart
\usepackage[default,defaultsans]{opensans}
\usepackage{qrcode}

\setlength{\parindent}{0pt}

\begin{document}
%%%%% BEGINN TITEL %%%%%
\newcommand*{\Autor}{}
\newcommand*{\Titel}{Hochschultransformation Jetzt!}
\newcommand*{\Untertitel}{Transformation zu mehr Nachhaltigkeit in der Hochschullehre}
\title{\Titel}

% Wir machen den Titel hier als großen Text, denn anders wird das nichts :(
\begin{center}
	%\large{\Untertitel}\\
	\vspace{0,1cm}
	\LARGE{\Titel}\\
	\large{\Untertitel}\\
\end{center}



%%%%% ENDE TITEL %%%%%

%%%%% BEGINN INHALT %%%%%

\hypertarget{pruxe4ambel}{%
\section*{Präambel}\label{pruxe4ambel}}

Wir sind Vertreter:innen unterschiedlicher Disziplinen und aller
Statusgruppen deutscher Hochschulen und Teilnehmende am Jahresprogramm
2022/2023 „Hochschullehre im Kontext von Nachhaltigkeit'' der Stiftung
Innovation in der Hochschullehre (StIL). Wir fühlen uns dem Konzept
\emph{Bildung für Nachhaltige Entwicklung} (BNE) verpflichtet. In diesem
Sinne fordern wir, das Hochschulsystem nachhaltig zu transformieren, um
die aktuellen und zukünftigen Herausforderungen bewältigen zu können. Um
dies erfolgreich zu verändern und zu gestalten, muss eine kritische
Masse von Akteur:innen diese Veränderung einfordern, gestalten,
mittragen und kooperativ umsetzen.

Hochschulen tragen die Verantwortung, gesellschaftliche Transformation
wissenschaftlich, kritisch, konstruktiv und progressiv zu unterstützen
und zu gestalten. Die im Handlungsfeld Lehre Tätigen haben damit die
Chance, Studierende auf Transformationsprozesse einer Nachhaltigen
Entwicklung vorzubereiten. Im Fokus steht dabei die Förderung
zukunftsorientierter Kompetenzen im Verständnis von BNE.

Als Studierende, wissenschaftlich Mitarbeitende, Professor:innen,
Lehrbeauftragte, Hochschuldidaktiker:innen, Verwaltungsmitarbeitende,
Hochschul-, Fakultäts- und Studiengangsleitungen haben wir vielfältige
Perspektiven auf Hochschullehre. Lehre im Kontext von Nachhaltigkeit
begreifen wir im Sinne eines \emph{Whole Institution Approach}, als
ganzheitlichen Bildungsauftrag aller Akteursgruppen.

Mit dieser Erklärung geben wir eine Richtung vor, die für die
strategischen, inhaltlichen und vor allem strukturellen Änderungen an
Hochschulen wegweisend ist. Ziel muss es sein, nicht nur das
Hochschulsystem zu transformieren, sondern darüber hinaus in die
Gesellschaft hineinzuwirken.

\hypertarget{dringlichkeit-nachhaltiger-entwicklung-in-der-hochschullehre}{%
\section{Dringlichkeit nachhaltiger Entwicklung in der
Hochschullehre}\label{dringlichkeit-nachhaltiger-entwicklung-in-der-hochschullehre}}

1.1 Vom \textbf{Ziel der Nachhaltigkeit} sind wir als Gesellschaft weit
entfernt. Daher muss das deutsche Hochschulsystem einen vehementeren
Beitrag zum \textbf{Prozess nachhaltiger Entwicklung} leisten.

1.2 Nachhaltige Transformation ist in Wissenschaft und Forschung ebenso
wie in Gesellschaft und Politik ein zentrales Konzept und hat einen Wert
auf lokaler, nationaler sowie auf internationaler Ebene. Sie beschreibt
den Prozess, der auf einen \textbf{Idealzustand menschlicher
Zivilisation abzielt, in dem gleichzeitig die ökologische Tragfähigkeit,
die soziale Gerechtigkeit und die wirtschaftliche Leistungsfähigkeit}
gewährleistet wird.

1.3 Ökologische Ressourcen werden
\href{https://www.stockholmresilience.org/research/planetary-boundaries.html}{stärker
beansprucht als diese sich regenerieren} können. Einerseits herrschen
\textbf{ungerechte Strukturen} und Defizite, andererseits führen falsche
Handlungsanreize zu nicht-nachhaltigen Entscheidungen und Entwicklungen.

1.4 Neben dem Klimawandel weisen \textbf{mehrere Krisen auf
Überschreitungen der ökologischen Grenzen hin}: Bei Diversitäts- und
Artenschwund, Landnutzung, lebensnotwendigen Stoffkreisläufen und der
unkontrollierten Verbreitung neuer Substanzen werden die planetaren
Grenzen dauerhaft überschritten. Der \textbf{irreversible Verlust}
unserer bisherigen Lebensgrundlagen betrifft alle Teilbereiche der
Zivilisation und löst akuten Transformationsdruck aus. Das Wissen um
langfristige Kopplungen, komplexe Auswirkungen und Kipppunkte in vielen
Systemen belegt, dass die Entscheidungs- und Handlungsspielräume
begrenzt und die angerichteten Schäden irreparabel sind.

1.5 Auch die \textbf{sozialen Grenzen} des sicheren und gerechten
Handlungsraums werden lokal und global für viele Menschen nicht
eingehalten. Die
\href{https://www-cdn.oxfam.org/s3fs-public/file_attachments/dp-a-safe-and-just-space-for-humanity-130212-en_5.pdf}{Defizite}
manifestieren sich sich in Form von Mangel und Versagen beim Sichern der
Menschenrechte und ungerechter Verteilung von Gütern manifestiert. Ohne
globale und intergenerationelle Gerechtigkeit wird Nachhaltigkeit
unerreicht bleiben.

1.6 Die \textbf{wissenschaftlichen Aspekte} von Nachhaltigkeit haben
sich in den letzten 50 Jahren in einem akademischen Diskurs
richtungsweisend etabliert. Dieser führt Erkenntnisse aus allen
Disziplinen zusammen und sucht im Austausch mit allen gesellschaftlichen
Akteuren nach Pfaden nachhaltiger Entwicklung. Damit bietet sich eine
fundierte theoretische Grundlage für anwendungsorientiertes Handeln.

\hypertarget{die-transformation-der-hochschullehre-gestalten}{%
\section{Die Transformation der Hochschullehre
gestalten}\label{die-transformation-der-hochschullehre-gestalten}}

2.1 \textbf{Studienangebote} sollten die Komplexität des Themas
angemessen abbilden und den Beitrag der jeweiligen Fachgebiete zur
Problemlösung reflektieren. Über die Dimensionen der Nachhaltigkeit des
Drei-Säulen-Konzepts (u.a. im
\href{https://sustainabledevelopment.un.org/content/documents/5987our-common-future.pdf}{Brundtlandbericht},
1987) hinaus zählen die von den Vereinten Nationen beschriebenen
Transformationsfelder und 17 \href{https://sdgs.un.org/goals}{Ziele
nachhaltiger Entwicklung}. Eine einseitige Orientierung, beispielsweise
allein auf technologische Innovationen hin, kann die Herausforderungen
nicht bewältigen. Der Anspruch von Hochschulbildung muss sich heute mehr
denn je auch auf die Persönlichkeits- und Wertebildung erstrecken.

2.2 \textbf{Alle Angehörigen der Hochschulen}, das heißt Studierende,
Dozierende, das wissenschaftliche Personal, Mitarbeitende im
wissenschaftsunterstützenden Bereich und in der Verwaltung sowie die
Hochschulleitungen, müssen \textbf{Verantwortung für den Umbau} der
Lehre übernehmen. Das kann nur gelingen, wenn allen Akteur:innen die
dazu notwendigen Handlungsspielräume eingeräumt werden und sie bei der
aktiven Nutzung dieser Spielräume unterstützt werden.

2.3 Nachhaltigkeit und Nachhaltige Entwicklung als \textbf{Lehrinhalte}
sowie der \textbf{Erwerb von Nachhaltigkeitskompetenzen} als
Qualifikationsziele müssen in geeigneter Form Eingang in alle Studien-
und Weiterbildungsangebote der Hochschulen finden. Dazu können je nach
fachlichem Hintergrund Zusatzangebote und Zertifikatsprogramme,
vollständige Studiengänge oder Schwerpunkte, spezifische Module oder
Inhalte einzelner Lehrveranstaltungen zum Themenfeld Nachhaltigkeit
zählen.

2.4 Da \textbf{kompetenz-, zukunfts- und wertorientierte Bildung} noch
nicht flächendeckend im Hochschulsystem umgesetzt wird, sind
entsprechende Transformationen notwendig. Alle im Handlungsfeld Lehre
Tätigen benötigen die finanziellen, zeitlichen, räumlichen,
vertraglichen und ideellen Ressourcen. Dies schließt die grundlegende
Qualifizierung, Handlungsempfehlungen und geeignete
Unterstützungsstrukturen ein. Die Professionalisierung der Arbeit für
nachhaltige Entwicklung erfordert nicht nur die Einrichtung
einschlägiger Professuren, sondern die breite Sensibilisierung und
fachliche wie didaktische \textbf{Weiterqualifizierung von Lehrenden}
aller Fächer.

2.5 Da die drängenden Fragen unserer Zeit nicht aus der Perspektive
einzelner Disziplinen zu lösen sind, sollte \textbf{Inter- und
Transdisziplinarität} strukturell gefördert werden, beispielsweise durch
Anreizsysteme für die Durchführung von inter- und transdisziplinären
Lehrveranstaltungen und die Anrechnung des Aufwandes in einem
angemessenen Umfang. Dazu zählen auch die Förderung und Anerkennung der
Zusammenarbeit zwischen Studierenden unterschiedlicher Disziplinen und
ihrer erbrachten Leistung.

2.6 \textbf{Studierende} sind stärker als bisher an der
Weiterentwicklung der Hochschullehre \textbf{zu beteiligen}, weil
Nachhaltigkeit eine gleichwertige Berücksichtigung aller Interessen
fordert. \textbf{Studierende} bringen sich in die Umgestaltung der Lehre
und des Lernorts Hochschule ein. Sie werden so in die Lage versetzt,
nicht nur Lernende zu sein, sondern aktiv und kompetent die
Transformation im Hochschulsystem und den gesellschaftlichen Wandel
mitzugestalten

\hypertarget{anforderungen-an-die-rahmenbedingungen}{%
\section{Anforderungen an die
Rahmenbedingungen}\label{anforderungen-an-die-rahmenbedingungen}}

3.1 Die \textbf{Hochschulleitungen} stehen zusammen mit den
\textbf{Landesregierungen} in der Verantwortung, für Lehrende und
Studierende die \textbf{Bedingungen für Lehre im Kontext Nachhaltiger
Entwicklung} zu schaffen. Dazu gehören Räume und deren Einrichtung, die
Interaktion, Kooperation und Umsetzung der erworbenen Kompetenzen
ermöglichen, sowie die entsprechende technische Infrastruktur.
Entsprechende Investitionen haben Priorität, bspw. in Reallabore und
Maker Spaces inkl. personeller Ausstattung für interdisziplinäre Lehre.
Verantwortliche für BNE-orientierte Lehre sind auf allen Ebenen der
Hochschule zu benennen. Ansprechpersonen zur Vernetzung innerhalb und
zwischen Hochschulen sind zu benennen und entsprechend auszustatten.

3.2 \textbf{Alle Lehrenden} sollen sich für aktuelle Diskurse zu
Nachhaltiger Entwicklung und BNE interessieren, um Anknüpfungspunkte für
Ihre Lehre zu erkennen. Ihre Einsatz für BNE- orientierte Lehre ist von
Vorgesetzten und Leitungen an Hochschulen anzuerkennen sowie
entsprechende Freiräume sind aufzutun. Lehrende sind bei der
Curriculumsentwicklung und der Umsetzung innovativer Lehrkonzepte zu
unterstützen

3.3 Jenseits der gesetzlich geregelten Beteiligung in der akademischen
Selbstverwaltung sind alle Hochschulen aufgerufen, \textbf{Studierenden}
Freiräume und Unterstützung bei der Umsetzung von Nachhaltigkeitszielen
einzuräumen und sie \textbf{umfassend in die Gestaltung der Hochschulen
als Räume für Nachhaltige Entwicklung einzubinden}.

3.4 Die \textbf{Gestaltung der Third Mission} (Transfer, Kooperation,
Kommunikation und Weiterbildung) ist neben Lehre und Forschung
\textbf{im Kontext Nachhaltiger Entwicklung} ein gleichwertiges Ziel.
Studierende, Lehrende und Hochschulleitung suchen den Kontakt zu
Nachhaltigkeitsakteur:innen aus anderen Bildungseinrichtungen und
Systemen, um neue Impulse für Lehre im Kontext Nachhaltiger Entwicklung
zu erhalten.

3.5 \textbf{Wissenschaftsunterstützende Maßnahmen} für Lehre im Kontext
Nachhaltiger Entwicklung sind auszuschreiben und mit ausreichenden
Ressourcen auszustatten.

3.6 Die Ergebnisse von Lehrforschungsprojekten zu Nachhaltigkeit und BNE
müssen über vielfältige Kanäle nach innen und außen an die
Öffentlichkeit kommuniziert werden. Hochschulleitung, Lehrende und
Studierende suchen den \textbf{Kontakt zu Nachhaltigkeitsakteur:innen}
aus anderen Bildungseinrichtungen, Politik und Verwaltung, Ökonomie und
Zivilgesellschaft für Impulse im Zusammenhang mit Bildung für
Nachhaltige Entwicklung.

3.7 \textbf{Partnerschaftliche Kooperationen und Austausch} mit
Lehrenden an anderen Hochschulen im In- und Ausland sollen stärker
gefördert werden, damit der Diskurs um eine globale Nachhaltigkeit
gerecht gestaltet und Erfahrungen mit BNE-orientierter Lehre
zielorientiert genutzt wird.

3.8 Die Umgestaltung der Lehre im Kontext der Nachhaltigen Entwicklung
darf nicht in Leitbildern und Strategiepapieren verschwinden. Sie muss
in Zielen und Maßnahmen konkretisiert und konsequent umgesetzt werden.
Zur Evaluation, Messung und Überprüfung sind Indikatoren und Instrumente
zu entwickeln. Ein \textbf{Qualitätsmanagementsystem}, das auf die
Nachhaltigkeitsziele ausgerichtet ist, unterstützt die Weiterentwicklung
der Lehre systematisch. Systemakkreditierte Hochschulen sind in der
Lage, ihre Qualitätssicherungsverfahren entsprechend anzupassen. Im
Rahmen der eigenen Akkreditierungsverfahren erfolgt dann eine
Überprüfung der Qualitätsziele, die in der Regel zu weiteren
Entwicklungsimpulsen der Lehre führen.

3.9 Die \textbf{Entwicklungen und Fortschritte in allen
Handlungsfeldern} der Hochschulen sollen regelmäßig dokumentiert und
öffentlich zugänglich gemacht werden.

\hypertarget{fazit}{%
\section{Fazit}\label{fazit}}

Unsere Hochschulen stehen vor der Herausforderung, eine nachhaltige und
klimagerechte Zukunft mitzugestalten. Dafür brauchen wir nicht nur
Wissen, sondern auch Kompetenzen, Haltungen und Werte, die uns
befähigen, verantwortungsvoll zu handeln und Veränderungen anzustoßen.
Bildung für Nachhaltige Entwicklung und Klimagerechtigkeit bieten uns
die Möglichkeit, systemische Zusammenhänge zu verstehen, kritisch zu
reflektieren und partizipativ zu gestalten. Wir alle sind gefragt,
unsere Handlungsspielräume zu nutzen und uns aktiv für eine
Transformation unserer Hochschulen einzusetzen. Es reicht nicht aus,
sich auf symbolische Maßnahmen zu beschränken oder abzuwarten, bis
andere handeln. Wir müssen jetzt handeln, um den ökologischen und
sozialen Herausforderungen unserer Zeit gerecht zu werden. Damit wir
aber Handeln können, müssen die systemischen Rahmenbedingungen es auch
ermöglichen.

\hypertarget{unterzeichnerinnen}{
\section*{Unterzeichner:innen}\label{unterzeichnerinnen}}

\href{https://hochschultransformation.jetzt/unterzeichnende.html}{\underline{Auf unserer Webseite}} Ist eine Liste der Unterzeichner:innen
und ihrer Afiiliationen zu finden.

\textbf{Entwurfshinweis: In der finalen Version wird hier eine sehr schöne, Unterschriften-Seite zu finden sein…}

\newpage

\hypertarget{weitere-themen-die-aktuell-diskutiert-werden1}{%
\section*{Weitere Themen, die aktuell diskutiert
werden:\footnote{Nachhaltige Entwicklung ist ein Prozess. In diesem Sinne
  haben wir „Hochschultransformation jetzt!'' und das Addendum
  entwickelt. „Hochschultransformation jetzt!'' drückt einen Konsens
  aus, das Addendum ist eine Momentaufnahme einzelner im Juni 2023
  laufenden Diskussionsprozesse.}}
 \label{weitere-themen-die-aktuell-diskutiert-werden1}}

Manche von uns fordern studentische Vizepräsident:innen an jeder
Hochschule.
Manche von uns fordern eine verpflichtende Weiterbildung der Lehrenden,
die ihnen ein komplexes und kritisches Verständnis von Nachhaltigkeit
vermittelt und sie befähigt, transdisziplinäre und transformative
Lehrformate zu gestalten.

Manche von uns fordern anzuerkennen, dass die zentrale These führender
Nachhaltigkeitsstrategien -- die Machbarkeit einer ausreichend schnellen
absoluten Entkopplung von Wirtschaftswachstum und Naturzerstörung im
Sinne des sog. „grünen Wachstums'' -- bislang weder empirisch
nachgewiesen noch überzeugend theoretisch begründet worden ist. Daher
fordern einige von uns, dass Hochschulen ein wachstumsunabhängiges
Wirtschafts- und Gesellschaftssystem einfordern, mitgestalten und sich
darauf ausrichten.

Manche von uns betonen die kulturell-historische Entwicklung (mit ihren
westlichen, weißen und kolonialen Traditionen) als Ursprünge für heutige
systemische Fehler. Insbesondere betonen sie die enge Verknüpfung
zwischen der Klimakrise und dem jahrhundertelangen Hegemonialanspruch
des westlichen, weißen Wissenschafts- und Wirtschaftssystems mit
Kontinuität bis in die Gegenwart. Während die Länder des globalen Südens
nur zu einem Bruchteil für die Entstehung der Klimakrise verantwortlich
sind, tragen sie schon jetzt die schwerwiegendsten Folgen mit der
zunehmenden Unbewohnbarkeit von Regionen und dem Verlust der
Lebensgrundlagen von Millionen Menschen.

Manche von uns fordern anzuerkennen, dass die Industrialisierung in den
letzten 200 Jahren Technologie und die Vorstellung, dass sich
gesellschaftliche Herausforderungen allein durch technologische
Innovationen bewältigen lassen, eine dominante Stellung in der
Gesellschaft verschafft hat. Manche von uns fordern daher ein, dass wir
uns für eine kritische Reflexion und Transformation der daraus
entstandenen Strukturen einsetzen.

Manche von uns stellen die Gefahr heraus, dass auch vielen
Wissenschaftler:innen, inklusive der Autor:innen, die Bedrohlichkeit der
Lage im Alltag immer wieder aus dem Blick zu geraten droht. Deshalb ist
es für manche von uns die besondere individuelle Verantwortung
(mehrfach) privilegierter Wissenschaftler:innen im privaten Bereich, als
Akteur:innen der Zivilgesellschaft, im privaten Konsumverhalten ebenso
wie in Forschung und Lehre alle ihnen verfügbaren Mittel einzusetzen, um
Politik und Zivilgesellschaft immer wieder an den Handlungsdruck zu
erinnern, sie zu radikaler Mitigation zu bewegen und dabei selbst eine
Vorbildrolle wahrzunehmen.

Manche von uns fordern anzuerkennen, dass die Klimakrise nicht isoliert
von Gerechtigkeitsaspekten wie Generationengerechtigkeit,
Geschlechtergerechtigkeit und anderen intersektional verknüpften
Dimensionen betrachtet und bewältigt werden kann, und plädieren daher
für eine systematische Auseinandersetzung mit diesem Aspekt.


\end{document}

